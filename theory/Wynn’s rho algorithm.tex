\documentclass[14pt, a4paper]{extarticle}
\usepackage[utf8]{inputenc}
\usepackage[T2A]{fontenc}
\usepackage[russian]{babel}
\usepackage{geometry}
\geometry{left=2cm, right=2cm, top=2cm, bottom=2cm}

% Увеличение межстрочного интервала
\usepackage{setspace}
\onehalfspacing % Полуторный интервал

% Настройки для равномерного распределения текста без переносов
\usepackage{microtype}
\frenchspacing
\sloppy
\hyphenpenalty=10000
\exhyphenpenalty=10000
\clubpenalty=10000
\widowpenalty=10000
\brokenpenalty=10000
\pretolerance=10000
\tolerance=10000
\emergencystretch=0.5em
\hbadness=10000

% Для правильного отображения математических формул
\usepackage{amsmath}
\usepackage{amssymb}

\usepackage{tocloft}
% Добавляем точку после номера раздела в оглавлении
\renewcommand{\cftsecaftersnum}{.}
\renewcommand{\cftsecleader}{\cftdotfill{\cftdotsep}} % Точки для разделов
\renewcommand{\cfttoctitlefont}{\hfill\Large\bfseries} % Центрирование заголовка без жирности
\renewcommand{\cftaftertoctitle}{\hfill} % Центрирование заголовка

% Убираем жирное начертание в содержании
\renewcommand{\cftsecfont}{\mdseries}        % Для разделов
\renewcommand{\cftsubsecfont}{\mdseries}     % Для подразделов
\renewcommand{\cftsubsubsecfont}{\mdseries}  % Для подподразделов
\renewcommand{\cftsecpagefont}{\mdseries}    % Для номеров страниц разделов
\renewcommand{\cftsubsecpagefont}{\mdseries} % Для номеров страниц подразделов
\renewcommand{\cftsubsubsecpagefont}{\mdseries} % Для номеров страниц подподразделов

\usepackage{graphicx}

% Изменяем формат сносок на верхний индекс
\makeatletter
\renewcommand{\@makefntext}[1]{%
  \noindent\textsuperscript{\@thefnmark}~#1%
}
\makeatother

\usepackage{fancyhdr}
\fancypagestyle{plain}{%
    \fancyhf{}
    \renewcommand{\headrulewidth}{0pt}
    \fancyfoot[C]{\thepage}
}
\pagestyle{fancy}
\fancyhf{}
\fancyfoot[R]{\thepage}
\renewcommand{\headrulewidth}{0pt}

\usepackage{indentfirst}
\setlength{\parindent}{1.25cm} % Размер отступа красной строки

% Для центрирования заголовка "Введение"
\usepackage{titlesec}
\titleformat{name=\section,numberless}[block]
    {\centering\Large\bfseries}{}{0pt}{}

% Переход на новую страницу для \section
\usepackage{titlesec}
\newcommand{\sectionbreak}{\clearpage}

% Теоремы и их нумерация
\usepackage{amsthm}

% Определения окружений
\newtheorem{theorem}{Теорема}[section] % нумерация внутри секции
\newtheorem{lemma}[theorem]{Лемма}      % будет идти как Теорема 1.2 и т.д.
\newtheorem{proposition}[theorem]{Утверждение}

\theoremstyle{definition}
\newtheorem{definition}[theorem]{Определение}

\theoremstyle{remark}
\newtheorem{remark}[theorem]{Замечание}

\usepackage{chngcntr}
% \counterwithout{subsection}{section}


\begin{document}

\pagenumbering{gobble} % Отключаем нумерацию страниц
\tableofcontents
\clearpage

\pagenumbering{arabic} % Включаем нумерацию
\setcounter{page}{2} % Начинаем с номера 2

\section{Введение}


Ускорение сходимости последовательностей и суммируемых рядов является
важной задачей в численных вычислениях. Применение специальных
алгоритмов и преобразований к рядам позволяет значительно сократить
количество итераций, необходимых для достижения желаемой точности,
сохраняя при этом значение суммы.

Одним из таких методов является $\rho$-алгоритм, разработанный Питером Винном
в 1956 году. Этот численный метод предназначен для ускорения сходимости
последовательностей, особенно чередующихся рядов. $\rho$-алгоритм основан на
использовании разностных схем и применяется для последовательностей,
сходящихся к пределу логарифмически. Его обобщения включают модификации
и расширения исходного алгоритма для улучшения эффективности и
применимости в различных ситуациях.

Другим важным методом является $\theta$-алгоритм, открытый Клодом Брезински в
1971 году. Данный метод также предназначен для ускорения сходимости и
является обобщением $\rho$-алгоритма Винна. $\theta$-алгоритм используется для
численного вычисления пределов и сумм бесконечных рядов и основан на
итерационном процессе, включающем различные шаги специальной
трансформации для улучшения скорости сходимости и повышения точности
расчетов.

Обобщенные версии $\rho$-алгоритма Винна и $\theta$-алгоритма Брезински могут
включать дополнительные модификации, такие как улучшенные стратегии
выбора параметров, оптимизированные процедуры вычислений и другие
методы, направленные на улучшение производительности и точности
численных вычислений.

\section{Основная терминология для последовательностей}

\subsection*{Обозначения множеств}
\(\mathbb{N}\) — множество натуральных чисел, \(\mathbb{N}=\{1,2,3,\dots\}\).

\(\mathbb{N}_0\) — множество натуральных чисел с нулём, \(\mathbb{N}_0=\mathbb{N}\cup\{0\}\).

\(\mathbb{R}\) — множество действительных чисел.

\(\mathbb{C}\) — множество комплексных чисел.

\subsection*{Последовательности и порядок сходимости}

Последовательность \(\{S_n\}\) — последовательность частичных сумм, где \(S_n\) определяется как сумма первых \(n\) членов последовательности \(\{a_n\}\), \(n\in\mathbb{N}_0\). Если \(n<0\), то \(S_n=0\).

\[
S_n=\sum_{j=0}^{n} a_j,\qquad n=0,1,\dots
\]

\(S=\sum_{j=0}^{\infty} a_j\) — предел последовательности частичных сумм (при существовании предела).

Определение порядка сходимости: последовательность \(\{x_n\}\), сходящаяся к \(S\), имеет порядок сходимости \(q\ge 1\) и скорость сходимости \(\mu\), если
\[
\lim_{n\to\infty}\frac{|x_{n+1}-S|}{|x_n-S|^q}=\mu.
\]

\subsection*{Асимптотическое поведение функций}

Пусть \(f(z)\) и \(g(z)\) — функции, определённые в области \(D\subset\mathbb{C}\), и пусть \(z_0\in D\). Тогда
\[
f(z)=O\bigl(g(z)\bigr),\quad z\to z_0
\]
означает, что существует константа \(A>0\) и окрестность \(U(z_0)\) такой, что для всех \(z\in U(z_0)\cap D\) выполняется
\[
|f(z)|\le A\,|g(z)|.
\]
Следствие: если \(g(z)\neq 0\) на \(U(z_0)\cap D\), то функция \(\dfrac{f(z)}{g(z)}\) ограничена на \(U(z_0)\cap D\).

Аналогично,\
$$
f(z)=o\bigl(g(z)\bigr),\quad z\to z_0
$$
означает, что для всякого \(\varepsilon>0\) существует окрестность \(U(z_0)\) такая, что для всех \(z\in U(z_0)\cap D\)
\[
|f(z)|\le \varepsilon\,|g(z)|.
\]
Следствие: если \(g(z)\neq 0\) на \(U(z_0)\cap D\), то \(\dfrac{f(z)}{g(z)}\to 0\) при \(z\to z_0\).

\subsection*{Асимптотические последовательности и разложения}

Последовательность функций \(\{\Phi_n(z)\}_{n\in\mathbb{N}_0}\), определённая в области \(D\subset\mathbb{C}\) и такая, что \(\Phi_n(z)\neq 0\) (кроме, возможно, в точке \(z_0\)), называется асимптотической последовательностью при \(z\to z_0\), если для всех \(n\in\mathbb{N}_0\)
\[
\Phi_{n+1}(z)=o\bigl(\Phi_n(z)\bigr),\quad z\to z_0.
\]

Формальный ряд \(f(z)\sim\sum_{n=0}^{\infty} c_n \Phi_n(z)\) называется асимптотическим разложением (в смысле Пуанкаре) относительно последовательности \(\{\Phi_n\}\), если для любого \(m\in\mathbb{N}_0\)
\[
f(z)-\sum_{n=0}^{m-1} c_n\Phi_n(z)=o\bigl(\Phi_m(z)\bigr),\quad z\to z_0.
\]
Если такое разложение существует, то оно единственно, и коэффициенты \(c_m\) могут быть вычислены по формуле
\[
c_m=\lim_{z\to z_0}\frac{f(z)-\sum_{n=0}^{m-1} c_n\Phi_n(z)}{\Phi_m(z)},\qquad m\in\mathbb{N}_0,
\]
при условии существования соответствующих пределов.

\subsection*{Сходящиеся и расходящиеся последовательности}

Если последовательность \({\{ S}_{n}\}\) сходится, то число, к которому
она стремится, называется пределом. В случае, когда последовательность
\({\{ S}_{n}\}\) расходится, число \(S\) называется антипределом, если
существует метод, позволяющий суммировать \({\{ S}_{n}\}\) к этому
значению. Значение антипредела зависит от характера расходящейся
последовательности, и поэтому точного определения для него нет.




\textbf{Важные утверждения о расходящихся последовательностях}

\begin{itemize}
\item
  Расходящиеся последовательности могут быть интерпретированы таким
  образом, что им можно сопоставить некоторые значения, называемые
  антипределами.
\item
  Для аппроксимации антипределов могут использоваться экстраполяционные
  методы, позволяющие оценить значения, к которым расходящиеся
  последовательности могли бы сходиться при определенных условиях.
\item
  Могут быть обработаны так же, как и сходящиеся, как с вычислительной,
  так и с теоретической точки зрения.
\end{itemize}

Примеры антипределов рядов
\begin{enumerate}
  \item Гармонический ряд натуральных чисел:
  \[
  1 + 2 + 3 + 4 + \dots = -\frac{1}{12},
  \]
  где антипредел получен через аналитическое продолжение дзета-функции Римана:  
  \(\zeta(-1) = -\frac{1}{12}\).

  \item Знакочередующийся ряд единиц:
  \[
  1 - 1 + 1 - 1 + \dots = \frac{1}{2},
  \]
  где антипредел вычисляется по методу Чезаро.

  \item Ряд
  \[
  1 - 2 + 3 - 4 + 5 - 6 + \dots = \frac{1}{4},
  \]
  антипредел которого вычисляется с помощью суммирования Абеля.
\end{enumerate}

Эти примеры показывают, что даже расходящиеся ряды могут иметь строго определённое значение в рамках теории суммирования, которое и называется их антипределом.



\subsection*{Остаток последовательности и его оценка}

Пусть \(\{S_n\}\) либо сходится к пределу \(S\), либо, если она расходится, может быть просуммирована подходящим методом для получения \(S\).

Тогда элемент последовательности \(S_n\) для всех \(n \in \mathbb{N}_0\) может быть представлен в виде суммы предела (или антипредела) \(S\) и остатка \(r_n\):
\[
S_n = S + r_n.
\]

Так как \(S_n\) — частичные суммы ряда
\[
S_n = \sum_{k=0}^{n} a_k,
\]
то остатки имеют вид
\[
r_n = - \sum_{k=n+1}^{\infty} a_k.
\]

Преобразования последовательностей различаются в зависимости от предположений о поведении остатков \(r_n\) как функций от \(n\). Эти предположения приводят к различным стратегиям частичного исключения остатков \(r_n\).

Пусть функция \(f(z)\) имеет асимптотическое разложение по асимптотической последовательности \(\{\Phi_n(z)\}_{n\in\mathbb{N}_0}\). Тогда первый член ряда \(\Phi_0(z)\) называется ведущим членом и обозначается
\[
f(z) \sim \Phi_0(z),
\]
что означает
\[
\frac{f(z)}{\Phi_0(z)} \to c_0 \quad \text{при } z \to z_0.
\]

В рассматриваемых трансформациях используются функции \(\omega_n\):
\[
\frac{r_n}{\omega_n} \sim \sum_{k=0}^{\infty} c_k \varphi_k(n), \quad n \to \infty,
\]
где \(\{\varphi_k(n)\}\) — подходящая асимптотическая последовательность.


\subsection*{Виды сходимости}

Поведение многих сходящихся последовательностей \({\{ S}_{n}\}\),
сходящихся к некоторому пределу \(S\) можно охарактеризовать
асимптотическим условием:

\[\lim_{n \rightarrow \infty}\frac{s_{n + 1} - s}{s_{n} - s} = \lim_{n \rightarrow \infty}\frac{r_{n + 1}}{r_{n}} = \rho\]

Последовательность \({\{ S}_{n}\}\) сходится:

\begin{itemize}
\item
  Линейно, если \(0 < |\rho| < 1\)
\item
  Логарифмически, если \(\rho = 1\)
\item
  Гиперлинейно, если \(\rho = 0\)
\end{itemize}

При \(|\rho| > 1\) последовательность расходится.

\subsection*{Класс \(F^{(m)}\)}

Мы говорим, что функция \(A(y)\), определённая для \(y \in (0,b]\) (\(b>0\)), где \(y\) — дискретная или непрерывная переменная, принадлежит множеству \(F^{(m)}\), \(m \in \mathbb{N}\), если существуют функции \(\phi_k(y)\) и \(\beta_k(y)\) (\(k=1,2,\dots,m\)) и константа \(A\), такие что
\[
A(y) = A + \sum_{k=1}^{m} \phi_k(y)\, \beta_k(y).
\]

Функции \(\phi_k(y)\) определены для \(y \in (0,b]\), а функции \(\beta_k(\xi)\), где \(\xi\) — непрерывная переменная, непрерывны на \([0,\xi_0]\) (\(\xi_0 \le b\)) и имеют асимптотическое разложение
\[
\beta_k(\xi) \sim \sum_{i=0}^{\infty} \beta_{ki}\, \xi^{i r_k}, \quad \text{при } \xi \to 0^+, \quad k=1,\dots,m, \quad r_k>0 \text{ — константы.}
\]

\textbf{Утверждение.} Пусть \(A_1(y) \in F^{(m_1)}\), предел или антипредел которой равен \(A_1\), и \(A_2(y) \in F^{(m_2)}\), предел или антипредел которой равен \(A_2\). Тогда функция
\[
A_1(y) + A_2(y) \in F^{(m)}, \quad m \le m_1 + m_2,
\]
и её предел или антипредел равен \(A_1 + A_2\).

\subsection*{Класс \(A_{0}^{(\gamma)}\)}

Функция $\alpha(x)$ определённая для сколь угодно больших $x > 0$,
принадлежит множеству \(A_{0}^{(\gamma)}\), если у неё есть
асимптотическое разложение формы:

\[\alpha(x)\sim x^{\gamma}\sum_{i = 0}^{\infty}\frac{\alpha_{i}}{x^{i}}\ \text{ при }\ x \rightarrow \infty,\ \ \gamma\ \epsilon\mathbb{\ C}\]

Если \(\alpha_{0} \neq 0\), то \(\alpha(x) \in \ A_{0}^{(\gamma)}\)
строго.

\subsection*{Класс b\textsuperscript{(m)}}

Последовательность \({\{ a}_{n}\}\) принадлежит множеству \(b^{(m)}\),
если она удовлетворяет линейному однородному разностному уравнению
порядка m:

\[a_{n} = \sum_{k = 1}^{m}{p_{k}(n)\Delta^{k}a_{n}}\]

\(p_{k}\ \epsilon{\ A}_{0}^{(k)}\) k = 1, \ldots, m так, что
\(p_{k}\ \epsilon\ A_{0}^{(i_{k})}\) строго для некоторого целого числа
\(i_{k} \leq k\).

Утверждение: Если \({\{ a}_{n}\} \in b^{(m)}\), тогда \({\{ a}_{n}\} \in b^{(q)}\) для каждого \(q > m\).

\subsection*{Классы последовательностей}

\begin{enumerate}
\item
  Логарифмически сходящиеся:

\(\left\{ S_{n} \right\} \in b^{(1)}/LOG\), если

\[
S_{n} \sim S 
+ n^{\gamma} \sum_{i = 0}^{\infty} \frac{\alpha_{i}}{n^{i}} \, 
\qquad \text{при } n \to \infty,\ \gamma \neq 0,1,\ldots,\ \alpha_{0} \neq 0.
\]

\item
  Линейно сходящиеся:

\(\left\{ S_{n} \right\} \in b^{(1)}/LIN\), если

\[S_{n}\sim S + \zeta^{n}n^{\gamma}\sum_{i = 0}^{\infty}\frac{\alpha_{i}}{n^{i}}\ \text{при } n \rightarrow \infty,\zeta \neq 1,\ \alpha_{0} \neq 0.\]

\item
  Факториально сходящиеся:

\(\{ S_{n} \} \in b^{(1)}/FAC\), если
\[
S_{n} \sim 
S + \frac{\zeta^{n} n^{\gamma}}{(n!)^{r}}
\sum_{i = 0}^{\infty} \frac{\alpha_{i}}{n^{i}}
\qquad \text{при } n \to \infty,\ r \neq 0,1,\ldots,\ \alpha_{0} \neq 0.
\]

\item
  Факториально расходящиеся:

\(\{ S_{n} \} \in b^{(1)}/FACD\), если
\[
S_{n} \sim 
(n!)^{r} \, \zeta^{n} \, n^{\gamma}
\sum_{i = 0}^{\infty} \frac{\alpha_{i}}{n^{i}}
\qquad \text{при } n \to \infty,\ r \in \mathbb{N}.
\]
\end{enumerate}

\subsection*{Преобразование последовательности}

Последовательность \{\(S_{n}\)\}, которая либо расходится, либо сходится
настолько медленно, что её применение становится практически
невозможным, преобразовывается с помощью функции \(T\) в новую
последовательность \{\(S_{n}^{'}\)\}, которая сходится быстрее:

\(T\):\{\(\ S_{n}\)\}→\{\(S_{n}^{'}\)\}, \(n \in \mathbb{N}_{0}\)

Вычислительные алгоритмы могут выполнять только конечное число операций,
поэтому будут работать лишь с конечными подмножествами
последовательностей, содержащими последовательные элементы \{\(S_{n}\),
\(S_{n + 1}\), ..., \(S_{n + l}\)\}, где \(l\) --- порядок
преобразования.

Преобразование \(T\) представляется как функция:

\(T\): \(\mathbb{R}^{l + 1}\) → \(\mathbb{R}\).

Каждое преобразование может быть записано в виде двумерной таблицы
\(T_{k}^{(n)}\), где верхний индекс \(n\) указывает строку, а нижний
индекс \(k\) --- столбец:

\[\begin{matrix}
T_{0}^{(0)} & T_{1}^{(0)} & T_{2}^{(0)} & \ldots & T_{n}^{(0)} & \ldots \\
T_{0}^{(1)} & T_{1}^{(1)} & T_{2}^{(1)} & \ldots & T_{n}^{(1)} & \ldots \\
T_{0}^{(2)} & T_{1}^{(2)} & T_{2}^{(2)} & \ldots & T_{n}^{(2)} & \ldots \\
 \vdots & \vdots & \vdots & \ddots & \vdots & \ddots \\
T_{0}^{(n)} & T_{1}^{(n)} & T_{2}^{(n)} & \ldots & T_{n}^{(2)} & \ldots \\
 \vdots & \vdots & \vdots & \ddots & \vdots & \ddots \\
\end{matrix}\]

Последовательность \(P\) = \{(\(n_{j}\), \(k_{j}\))\} упорядоченных пар
целых чисел \(n_{j}\),\(\ k_{j} \in \mathbb{N}_{0}\) называется путем, если \(n_{0} = k_{0} = 0\) и для всех
\(j\  \in \mathbb{N}_{0}\) выполняется \(n_{j + 1} \geq \ n_{j}\) и
\(k_{j + 1} \geq \ k_{j}\), причем хотя бы одно из отношений
\(n_{j + 1} = n_{j} + 1\) и \(k_{j + 1} = k_{j} + 1\) должно быть
истинным.

Преобразование \(T\) является регулярным на пути \(P\), если для любой
сходящейся последовательности \{\(S_{n}\)\} выполняется:

\[\lim_{j \rightarrow \infty}T_{k_{j}}^{{(n}_{j})} = S\]

Функция \(T\) называется ускоряющей сходимость, если:

\[\lim_{j \rightarrow \infty}\frac{T_{k_{j}}^{{(n}_{j})} - S}{S_{n_{j}} - S} = 0\]

Иначе говоря, \(T\) ускоряет сходимость последовательности
\(\{ S_{n}\}\) при преобразовании в \(\{ S_{n}^{'}\}\), если
\(\{ S_{n}^{'}\}\) сходится к \(S\) быстрее, чем \(\{ S_{n}\}\), то
есть:

\[\lim_{n \rightarrow \infty}\frac{\left| S_{n}^{'} - S \right|}{\left| S_{n} - S \right|} = 0\]

\subsection*{Символ Похгаммера}

Пусть \(\Omega(z)\) -- функция, стремящаяся к нулю при \(z \to\infty \) .
Факториальный ряд для \(\Omega(z)\) представляет собой разложение
следующего типа:

\[\Omega(z)\  = \frac{b_{0}}{z} + \frac{1!b_{1}}{z(z + 1)} + \frac{2!b_{2}}{z(z + 1)(z + 2)} + \ldots = \sum_{v = 0}^{\infty}\frac{v!b_{v}}{{(z)}_{v + 1}}\]

Символы Похгаммера (растущие факториалы) выражаются операцией

\[{(z)}_{v + 1} = \frac{\Gamma(z + v + 1)}{\Gamma(z)} = z(z + 1)\ldots(z + v)\]

В общем случае \(\Omega(z)\) будет иметь простые полюса в точках \(z =
-m\), где \(m \in \mathbb{N}_{0}\)


% \section{Процесс экстраполяции
% Ричардсона}

\section{Методы преобразования
последовательностей}


\subsection{\(\rho\) -- алгоритм Винна и обобщения}

\subsubsection*{\(\rho\) — алгоритм Винна}

Алгоритм \(\rho\)-Винна предназначен для вычисления чётных сходящихся
интерполирующих дробей Тиле и их экстраполяции к бесконечности.

Интерполирующая дробь Тиле, или чётная сходящаяся дробь, имеет вид
рациональной функции:

\[
\mathcal{S}_{2k}(x) =
\frac{a_{k}x^{k} + a_{k - 1}x^{k - 1} + \ldots + a_{1}x + a_{0}}
     {b_{k}x^{k} + b_{k - 1}x^{k - 1} + \ldots + b_{1}x + b_{0}},
\quad k \in \mathbb{N}_{0}.
\]

Здесь отношение \(\frac{a_{k}}{b_{k}}\) представляет собой приближение к
пределу. Чётные порядки конвергентов являются рациональными функциями,
представленными в виде частного двух полиномов. Алгоритм Винна позволяет
вычислять интерполирующую рациональную функцию и её экстраполяцию к
бесконечности с меньшим числом арифметических операций по сравнению с
аналогичными рекурсивными алгоритмами.

Метод \(\rho\) ускоряет сходимость логарифмических последовательностей в
\(b^{(1)}/\log\) и особенно эффективен для последовательностей
\(\{ S_{n} \}\), таких что

\[
S_{n} \sim S + \sum_{i = 1}^{\infty} \delta_{i} n^{-i},
\qquad n \to \infty.
\]

Поскольку \(S_{n} = h(n) \in A_{0}^{(0)}\),
\(h(n)\) ведёт себя плавно при \(n \to \infty\).
Следовательно, вблизи \(n = \infty\) функцию \(h(n)\)
можно эффективно аппроксимировать рациональной функцией \(R(n)\),
у которой степень числителя равна степени знаменателя.
Тогда \(\lim_{n \to \infty} R(n)\) может служить хорошим приближением для

\[
S = \lim_{n \to \infty} h(n) = \lim_{n \to \infty} S_{n}.
\]

В частности, функцию \(R(n)\) можно подобрать так, чтобы она интерполировала
\(h(n)\) в \(2k+1\) точках.

Как указали Смит и Форд, \(\rho\)-алгоритм Винна хорошо работает с
некоторыми логарифмическими последовательностями, но не работает с
другими, что и требует отдельного пояснения.

Процесс экстраполяции Ричардсона состоит в пропускании интерполяционного
полинома степени \(k\) через \(k + 1\) пар \((x_{n}, S_{n}), \ldots,
(x_{n+k}, S_{n+k})\) с использованием формулы Невилла–Эйткена,
затем вычисляют значение этого полинома при \(x = 0\).

Алгоритм \(\rho\) состоит из построения рациональной интерполяционной
дроби, числитель и знаменатель которой являются многочленами степени \(k\),
по \(2k+1\) парам точек \((x_{n}, S_{n}), \ldots, (x_{n+2k}, S_{n+2k})\)
с использованием интерполяционной формулы Тиле, а затем вычисления значения
этой рациональной дроби при \(x\).

Поскольку \(\rho\)-алгоритм — частный случай \emph{взаимных разностей},
начнём с их определения. Пусть \(f(x)\) — функция, взаимные разности которой
с аргументами \(x_{0}, x_{1}, \ldots\) определяются рекурсивно:

\[
\rho_{0}(x_{0}) = f(x_{0}).
\]

\[
\rho_{1}(x_{0},x_{1})
= \frac{x_{0}-x_{1}}{\rho_{0}(x_{0}) - \rho_{0}(x_{1})}.
\]

\[
\rho_{k}(x_{0},\ldots,x_{k})
=
\rho_{k-2}(x_{1},\ldots,x_{k-1})
+
\frac{x_{0}-x_{k}}{\rho_{k-2}(x_{1},\ldots,x_{k-1}) - \rho_{k-1}(x_{1}, \ldots,x_{k-1})},
\qquad k = 2,3,\ldots
\tag{2c}
\]

Заменив \(x_{0}\) на \(x\) в \((2c)\), получим следующую цепную дробь:

\[
f(x) =
f(x_{1})
+ \frac{x-x_{1}}{\rho_{1}(x_{1},x_{2}) +
\frac{x-x_{2}}{\rho_{2}(x_{1},x_{2},x_{3}) - \rho_{0}(x_{1}) +
\frac{x-x_{3}}{\ddots}}}.
\]

Последние две составляющие простейшие дроби в данной формуле цепной дроби имеют следующий вид:

\[
\frac{
  x - x_{l-1}
}{
  \rho_{l-1}(x_{1},\ldots,x_{l}) - \rho_{l-3}(x_{1},\ldots,x_{l-2}) 
  + 
  \frac{x - x_{l}}{
    \rho_{l}(x,x_{1},\ldots,x_{l}) - \rho_{l-2}(x_{1},\ldots,x_{l-1})
  }
}
\]

Равенство \((3)\) справедливо при \(x = x_{1},\ldots,x_{l}\).  
Правая часть равенства \((3)\) называется \textbf{интерполяционной формулой Тиля}.

Рассмотрим функцию \(f(x_{n})\), значение \(S_{n}\) которой известно в некотором числе точек \(x_{n}\), \(n \in \mathbb{N}_{0}\). $\rho$-алгоритм Винна определяется заменой \(S_{n}\) вместо \(f(x_{n})\) и \(\rho_{k}^{(n)}\) вместо \(\rho_{k}(x_{n},\ldots,x_{n+k})\) во взаимной разности:

\begin{subequations}
\begin{align}
\rho_{0}^{(n)} = S_{n},

\rho_{1}^{(n)} = \frac{x_{n} - x_{n+1}}{\rho_{0}^{(n)} - \rho_{0}^{(n+1)}}, 


\rho_{2}^{(n)} = \rho_{0}^{(n+1)} + \frac{x_{n} - x_{n+2}}{\rho_{1}^{(n)} - \rho_{1}^{(n+1)}},


\rho_{k}^{(n)} = \rho_{k-2}^{(n+1)} + \frac{x_{n} - x_{n+k}}{\rho_{k-1}^{(n)} - \rho_{k-1}^{(n+1)}}, \quad k = 2,3,\ldots
\end{align}
\end{subequations}


Покажем, что рациональная дробь \(R(x)\), числитель и знаменатель которой являются полиномами степени \(k\) и такая, что

\[
R(x_{p}) = S_{p}, \quad \forall p = n,\ldots,n+2k, 
\]

может быть записана в виде:

\[
R(x) = \frac{\rho_{2k}^{(n)} x^{k} + a_{1} x^{k-1} + \ldots + a_{k}}{x^{k} + b_{1} x^{k-1} + \ldots + b_{k}},
\]

Тогда получаем, что 

\[
\lim_{n \to \infty} R(x) = \rho_{2k}^{(n)},
\]

что позволяет использовать величину \(\rho_{2k}^{(n)}\) как приближение предела последовательности \(\{ S_{n} \}\) при \(n \to \infty\). Расчёт \(\rho_{2k}^{(n)}\) осуществляется с использованием расширенной формы \emph{$\rho$}-алгоритма, который по сути является расчётом взаимных разностей.


Нелинейная рекурсивная стандартная схема алгоритма \emph{$\rho$} Винна выглядит следующим образом:

\[
\rho_{k+1}^{(n)} = \rho_{k-1}^{(n+1)} + \frac{x_{n+k+1} - x_n}{\rho_{k}^{(n+1)} - \rho_{k}^{(n)}}, \quad k,n \in \mathbb{N}_{0},
\]

при этом учитывается, что 

\[
\rho_{-1}^{(n)} = 0, \quad \rho_0^{(n)} = S_n, \quad n \in \mathbb{N}_{0}.
\]

Данный метод работает с последовательностью строго возрастающих и неограниченных с ростом \(n\) интерполяционных точек \(\{x_n\}\), которые должны быть положительными и различными для всех \(n \in \mathbb{N}_{0}\):

\[
0 < x_0 < x_1 < x_2 < \ldots < x_m < x_{m+1} < \ldots, \quad \lim_{n \to \infty} x_n = \infty.
\]

Видно, что структура $\rho$-алгоритма идентична структуре $\epsilon$-алгоритма
Винна, но отличается наличием самой последовательности интерполяционных
точек. Только элементы c четным порядком \(\rho_{2k}^{(n)}\) в методе
\emph{$\rho$} используются для аппроксимации предела, тогда как элементы
\(\rho_{2k + 1}^{(n)}\) нечетного порядка служат вспомогательными
величинами и могут расходиться, если вcя последовательность сходится, то
есть величины с нечетным нижним индексом являются лишь промежуточными
расчетами и не имеют никакого значения.

Несмотря на формальное сходство, алгоритмы Винна \(\epsilon\) и \emph{$\rho$}
существенно различаются по способности ускорять сходимость. Алгоритм
\emph{$\rho$} Винна эффективен для логарифмически сходящихся
последовательностей, но не подходит для линейно сходящихся или
расходящихся последовательностей, в случае которых выгоднее будет
применять \(\epsilon\) алгоритм.

Поскольку дроби четного порядка \(\mathcal{S}_{2k}(x)\) интерполяционной цепной дроби построены таким образом, что они удовлетворяют условиям
интерполяции \(2k + 1\), то

Поскольку дроби четного порядка \(\mathcal{S}_{2k}(x)\) интерполяционной цепной дроби построены так, что они удовлетворяют условиям интерполяции в \(2k + 1\) точках, имеем:

\[
\mathcal{S}_{2k}\left(x_{n+j}\right) = S_{n+j}, \quad 0 \leq j \leq 2k. 
\]


\begin{theorem}
Если применить \emph{$\rho$}-алгоритм к последовательности \(\{S_n\}\), такой что
\[
S_n = \frac{S x_n^k + a_1 x_n^{k-1} + \dots + a_k}{x_n^k + b_1 x_n^{k-1} + \dots + b_k},
\]
то преобразование \(\rho_{2k}^{(n)} = S\) для всех \(n \in \mathbb{N}_0\).

\begin{proof}
Покажем верность утверждения с помощью интерполирующей дроби Тиля, которая имеет вид цепной дроби:

\[
S_n = S_m + \cfrac{n-m}{\rho_1^{(m)} + \cfrac{n-m-1}{\rho_2^{(m)} - \rho_0^{(m)} + \cfrac{n-m-2}{\rho_3^{(m)} - \rho_1^{(m)} + \ddots}}}
\]

Учитывая значение для \(2k+1\) дроби в цепочке Тиля:

\[
\cfrac{n-m-2k}{\rho_{2k+1}^{(m)} - \rho_{2k-1}^{(m)} + \cfrac{n-m-2k-1}{\ddots}}
\]

получаем:

\[
S_n = \frac{\rho_{2k}^{(m)} n^k + a_1 n^{k-1} + \dots + a_k}{n^k + b_1 n^{k-1} + \dots + b_k}.
\]

Следовательно, \(\rho_{2k}^{(m)} = S\) для всех \(m \in \mathbb{N}_0\). Таким образом, \(\rho\)-алгоритм представляет собой рациональную экстраполяцию, точную на последовательности, удовлетворяющей условию выше.
\end{proof}
\end{theorem}

\subsection*{Свойства \emph{$\rho$-}алгоритма}

Некоторые свойства \emph{$\rho$-} и \(\epsilon\)-алгоритмов Винна схожи.

\textbf{Свойство 1.}  

\[\rho_{2k}^{(n)} = \frac{\left| \begin{matrix}
1 & S_{n} & x_{n} & {x_{n}S}_{n} & \cdots & x_{n}^{k - 1} & x_{n}^{k - 1}S_{n} & {x_{n}^{k}S}_{n} \\
 \vdots & \vdots & \vdots & \vdots & \vdots & \vdots & \vdots & \vdots \\
1 & S_{n + 2k} & x_{n + 2k} & {x_{n + 2k}S}_{n + 2k} & \cdots & x_{n + 2k}^{k - 1} & x_{n + 2k}^{k - 1}S_{n + 2k} & x_{n + 2k}^{k}S_{n + 2k} \\
\end{matrix} \right|}{\left| \begin{matrix}
1 & S_{n} & x_{n} & {x_{n}S}_{n} & \cdots & x_{n}^{k - 1} & x_{n}^{k - 1}S_{n} & {x_{n}^{k}S}_{n} \\
 \vdots & \vdots & \vdots & \vdots & \vdots & \vdots & \vdots & \vdots \\
1 & S_{n + 2k} & x_{n + 2k} & {x_{n + 2k}S}_{n + 2k} & \cdots & x_{n + 2k}^{k - 1} & x_{n + 2k}^{k - 1}S_{n + 2k} & x_{n + 2k}^{k} \\
\end{matrix} \right|}\ \]

\textbf{Свойство 2 (Алгебраические).}  

\begin{enumerate}
    \item Если применение \emph{$\rho$-}алгоритма к последовательностям \(\{ S_{n} \}\) и \(\{ a S_{n} + b \}\) даёт соответственно значения \(\rho_{k}^{(n)}\) и \(\overline{\rho}_{k}^{(n)}\), то выполняется:
    \[
        \overline{\rho}_{2k}^{(n)} = a \, \rho_{2k}^{(n)} + b, \qquad
        \overline{\rho}_{2k + 1}^{(n)} = \frac{\rho_{2k}^{(n)}}{a}.
    \]
    
    \item Если применение \emph{$\rho$-}алгоритма к последовательностям \(\{ S_{n} \}\) и \(\left\{ \frac{a S_{n} + b}{c S_{n} + d} \right\}\) даёт соответственно значения \(\rho_{k}^{(n)}\) и \(\overline{\rho}_{k}^{(n)}\), то выполняется:
    \[
        \overline{\rho}_{2k}^{(n)} = \frac{a \, \rho_{2k}^{(n)} + b}{c \, \rho_{2k}^{(n)} + d}.
    \]
\end{enumerate}

Итак, \(\rho\)-алгоритм представляет собой рациональную экстраполяцию, точную на последовательности, имеющей асимптотическое разложение вида:

\[
    S_n \sim S + n^{\theta} \left( c_0 + \frac{c_1}{n} + \frac{c_2}{n^2} + \dots \right), \quad n \rightarrow \infty,
\]

где \(\theta\) — отрицательное целое число, а \(c_j\) — константы, не зависящие от \(n\).

\subsection*{Асимптотическое поведение $\rho$-алгоритма}

Чтобы описать асимптотическое поведение $\rho$ алгоритма, мы будем использовать следующую последовательность. Для заданного нецелого числа $\theta$ и заданного ненулевого действительного числа $c$ мы определяем последовательность $(C_n)$ следующим образом:

\begin{subequations}
\begin{align}
C_{-1} &= 0, \\
C_0 &= c, \\
C_{2k-1} &= C_{2k-3} + \frac{2k - 1}{\theta C_{2k-2}}, \quad k = 1, 2, \dots, \\
C_{2k} &= C_{2k-2} + \frac{2k}{(1 - \theta) C_{2k-1}}, \quad k = 1, 2, \dots
\end{align}
\end{subequations}

Эта последовательность $(C_n)$ называется \emph{ассоциированной последовательностью $\rho$ алгоритма относительно $\theta$ и $c$}.
Асимптотическое поведение $\rho$ алгоритма и ассоциированную последовательность связывают следующие 2 формулы
\begin{equation}
\frac{\rho_{2 k}^{(n)}-s}{s_{n+2 k}-s} \sim C_{2 k} \quad \text { при } n \rightarrow \infty, \theta \not \in \mathbb{Z}\
\end{equation} 

\begin{equation}
\rho_{2 k}^{(n)}=s+O\left((n+k)^{-k-2}\right), \quad \text { при } n \rightarrow \infty, \theta  \in \mathbb{Z}
\end{equation}


Для ассоциированной последовательности $\rho$ алгоритма справедливы следующие две теоремы.

\begin{theorem}
В соответствии с приведенной выше нотацией,
\begin{subequations}
\begin{align}
C_{2k-1} &= \frac{k (2 - \theta) (3 - \theta) \cdots (k - \theta)}{c \, \theta (1 + \theta) \cdots (k - 1 + \theta)}, \quad k = 1, 2, \dots, \\
C_{2k} &= \frac{c (1 + \theta) \cdots (k + \theta)}{(1 - \theta) (2 - \theta) \cdots (k - \theta)}, \quad k = 1, 2, \dots
\end{align}
\end{subequations}

\begin{proof}
Индукция по $k$. Для $k=1, C_{1}=C_{-1}+1 / c \theta=1 / c \theta, C_{2}=C_{0}+2 /(1-\theta) C_{1}= c(1+\theta) /(1-\theta)$. Предположим, что они верны для $k>1$. По предположению индукции имеем

\begin{subequations}
\begin{align}
C_{2 k+1} & =C_{2 k-1}+\frac{2 k+1}{\theta C_{2 k}} \\
& =\frac{k(2-\theta) \cdots(k-\theta)}{c \theta(1+\theta) \cdots(k-1+\theta)}+\frac{(2 k+1)(1-\theta) \cdots(k-\theta)}{c \theta(1+\theta) \cdots(k+\theta)} \\
& =\frac{(k+1)(2-\theta) \cdots(k-\theta)(k+1-\theta)}{c \theta(1+\theta) \cdots(k+\theta)} 
\end{align}
\end{subequations}


Аналогично,


\begin{equation}
C_{2 k+2}=\frac{c(1+\theta)(2+\theta) \cdots(k+1+\theta)}{(1-\theta) \cdots(k+1-\theta)} .
\end{equation}

\end{proof}

\end{theorem}

Заметим, что теорема 3.2 остается верной, когда $\theta$ является целым числом и $k<|\theta|$.\\
\begin{theorem}(Об асимптотике ассоциированной последовательности)
\begin{equation}
\lim _{k \rightarrow \infty} \frac{C_{2 k}}{k^{2 \theta}}=-\frac{c \Gamma(-\theta)} {\Gamma(\theta)},
\end{equation}


где $\Gamma(x)$ — гамма-функция.\\
Доказательство. С помощью предельной формулы Эйлера для гамма-функции


\begin{equation}
\Gamma(x)=\lim _{k \rightarrow \infty} \frac{k!k^{x}}{x(x+1) \cdots(x+k)}
\end{equation}




\end{theorem}


Теперь мы имеем асимптотическое поведение алгоритма $\rho$.\\
\begin{theorem}
    
 Пусть ( $S_{n}$ ) — последовательность, удовлетворяющая


\begin{equation}
S_{n} \sim S+n^{\theta}\left(c_{0}+\frac{c_{1}}{n}+\frac{c_{2}}{n^{2}}+\ldots\right), \quad \text { при } n \rightarrow \infty .
\end{equation}


Пусть ($C_{n}$) — ассоциированная последовательность алгоритма $\rho$ относительно $\theta$ и $c_{0}$ в (9). Пусть $A=(1-\theta)\left(-1 / 2+c_{1} / c_{0} \theta\right)$. Тогда справедливы следующие формулы.\\

\begin{equation}
\rho_{1}^{(n)}=C_{1}(n+1)^{1-\theta}\left[1+\frac{A}{n+1}+\frac{B_{1}}{(n+1)^{2}}+O\left((n+1)^{-3}\right)\right]
\end{equation}


где


\begin{equation}
B_{1}=\frac{\theta^{2}-1}{12}+\frac{c_{1}(1-\theta)}{2 c_{0}}+\frac{(1-\theta)^{2} c_{1}^{2}}{c_{0}^ {2} \theta^{2}}+\frac{c_{2}(2-\theta)}{c_{0} \theta} .
\end{equation}


\begin{equation}
\rho_{2}^{(n)}=s+C_{2}(n+1)^{\theta}\left[1+\frac{c_{1}}{c_{0}(n+1)}+\frac{B_{2}}{(n+1)^{2}}+O\left((n+1)^{-3}\right)\right],
\end{equation}

где

\begin{equation}
B_{2}=-\frac{c_{0} \theta(1+\theta)}{6(1-\theta)}+\frac{2 c_{1}^{2}}{c_{0} \theta(1-\theta)}+\frac{c_{2}\left(5-\theta^{2}\right)}{(1-\theta)^{2}} .
\end{equation}


Предположим, что $\theta \neq-1, \ldots, 1-k$. Для $j=1, \ldots, k$,

\begin{subequations}
    
\begin{align}
\rho_{2 j-1}^{(n)} & =C_{2 j-1}(n+j)^{1-\theta}\left[1+\frac{A}{n+j}\right]+O\left((n+j)^{-1-\theta}\right)  \\
\rho_{2 j}^{(n)} & =s+C_{2 j}(n+j)^{\theta}\left[1+\frac{c_{1}}{c_{0}(n+j)}\right]+O\left((n+j)^{\theta-2}\right) 
\end{align}
\end{subequations}


\begin{proof}
    
(10) Используя биномиальное разложение, получаем

\begin{subequations}
\begin{align}
& s_{n+1}-s_{n} \\
& =c_{0} \theta(n+1)^{\theta-1}\left[1-\frac{A}{n+1}+\left(-\frac{1-\theta}{6}+\frac{c_{1}(1-\theta)}{2 c_{0} \theta} +\frac{c_{2}}{c_{0} \theta}\right) \frac{\theta-2}{(n+1)^{2}}\right] \\
& \quad+O\left((n+1)^{\theta-3}\right)
\end{align}
    
\end{subequations}


Следовательно, получаем


\begin{equation}
\rho_{1}^{(n)}=C_{1}(n+1)^{1-\theta}\left[1+\frac{A}{n+1}+\frac{B_{1}}{(n+1)^{2}}+O\left((n+1)^{-3}\right)\right]
\end{equation}


(12), (14). Аналогично (10).

По теореме 3.4, когда $\theta$ в (9) не является целым числом, для фиксированного $k$,
\end{proof}

\begin{equation}
\frac{\rho_{2 k}^{(n)}-s}{s_{n+2 k}-s} \sim C_{2 k} \quad \text { при } n \rightarrow \infty 
\end{equation}


Когда $\theta$ является отрицательным целым числом, а именно $-k$, мы имеем $C_{0} \neq 0, \ldots, C_{2 k-2} \neq 0$ и $C_{2 k}=0$. Таким образом, из теоремы 3.4 следует, что


\begin{equation}
\rho_{2 k}^{(n)}=s+O\left((n+k)^{-k-2}\right), \quad \text { при } n \rightarrow \infty . 
\end{equation}

\end{theorem}
\uline{Примечание}:

\emph{$\rho$}-алгоритм --- алгоритм экстраполяции рациональной дроби,
числитель и знаменатель которой имеют одинаковую степень. Можно
рассматривать это как частный случай метода Булирша и Стоера, где
степени числителя и знаменателя произвольны.

\subsection{Причины применения $\rho$-алгоритма Винна для логарифмически
сходящихся рядов:}

\begin{enumerate}
\item \textbf{Преобразование интерполяционных точек}

Алгоритм \emph{$\rho$} Винна включает последовательность интерполяционных
точек \(\{ x_{n}\}\), что позволяет более гибко подходить к обработке
ряда. Логарифмически сходящиеся ряды характеризуются тем, что их члены
уменьшаются медленно, и традиционные методы ускорения сходимости могут
оказаться неэффективными. Интерполяционные точки дают возможность
алгоритму адаптироваться к медленной сходимости, обеспечивая более
точное аппроксимирование предела.

  \item \textbf{Адаптация к логарифмической сходимости}

\emph{$\rho$}-алгоритм Винна строит последовательность рациональных функций,
которая учитывает форму логарифмически сходящихся рядов. Он использует
четные порядки элементов \(\rho_{2k}^{(n)}\) для аппроксимации предела,
что позволяет лучше учитывать особенности поведения логарифмически
сходящихся рядов.

  \item \textbf{Комплементарные свойства}

Алгоритм \emph{$\rho$} Винна дополняет \(\epsilon\) алгоритм Винна, который эффективен
для линейно сходящихся последовательностей, но не может ускорить
логарифмическую сходимость. В то время как \(\epsilon\) алгоритм эффективен для
суммирования чередующихся расходящихся рядов, алгоритм \emph{$\rho$} Винна
специально разработан для работы с логарифмически сходящимися рядами,
что делает его эффективным инструментом в таких случаях.
\item \textbf{Устойчивость к осцилляциям и расходимости}

Логарифмически сходящиеся ряды часто не демонстрируют осцилляционного
поведения, характерного для некоторых других типов рядов.
\emph{$\rho$}-алгоритм Винна, учитывая свою структуру и использование
интерполяционных точек, обеспечивает устойчивость к осцилляциям и
помогает избежать расходимости, эффективно аппроксимируя пределы таких
рядов.
\end{enumerate}

\subsection*{Модификации $\rho$ -- алгоритма}

\subsection{$\theta$ -- алгоритм Брезински}

\[\vartheta_{- 1}^{(n)} = 0,\ \ \vartheta_{0}^{(n)} = S_{n},\ \]

\[\vartheta_{2k + 1}^{(n)} = \vartheta_{2k - 1}^{(n + 1)} + \ \frac{1}{\Delta\vartheta_{2k}^{(n)}},\ \ \]

\[\vartheta_{2k + 2}^{(n)} = \vartheta_{2k}^{(n + 1)} + \ \frac{\lbrack\Delta\vartheta_{2k}^{(n + 1)}\rbrack\lbrack\Delta\vartheta_{2k + 1}^{(n + 1)}\rbrack}{\Delta^{2}\vartheta_{2k + 1}^{(n)}},\ \ k,n = 0,\ 1,\ \ldots\]

\section{Заключение}

\section{Список
литературы}

\begin{enumerate}
\def\labelenumi{\arabic{enumi}.}
\item
  Brezinski, C. (1977). \emph{Acceleration de la Convergence en Analyse
  Numerique}. Springer-Verlag.
\item
  Osada, Naoki. \emph{Acceleration Methods for Slowly Convergent
  Sequences and Their Applications}. January 1993.
\item
  Weniger, E. J. (2003). Nonlinear Sequence Transformations for the
  Acceleration of Convergence and the Summation of Divergent Series.
  \emph{Computer Physics Repor}ts, 1(1), 1-123.
\item
  Brezinski, C., \& Redivo Zaglia, M. (2003). \emph{Extrapolation
  Methods: Theory and Practice}. Amsterdam: North-Holland.
\item
  Sidi, A. (2003). \emph{Practical Extrapolation Methods: Theory and
  Applications}. Cambridge: Cambridge University Press.
\item
  Van Tuyl, A. H. (1994). Acceleration of Convergence of a Family of
  Logarithmically Convergent Sequences. \emph{Mathematics of
  Computation}, 63(207), 229-246. American Mathematical Society.
\item
  Weniger, E. J. (1990). On the derivation of iterated sequence
  transformations for the acceleration of convergence and the summation
  of divergent series. \emph{Institut für Physikalische und Theoretische
  Chemie, Universität Regensburg}, W-8400 Regensburg, Germany.
\item
  Borghi, R., \& Weniger, E. J. (2015). Convergence analysis of the
  summation of the factorially divergent Euler series by Padé
  approximants and the delta transformation. \emph{Dipartimento di
  Ingegneria, Università "Roma Tre", I-00144 Rome, Italy and Institut
  für Physikalische und Theoretische Chemie, Universität Regensburg,
  D-93040 Regensburg, Germany.}
\end{enumerate}

\end{document}